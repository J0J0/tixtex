\documentclass{tixtex}
\hypersetup{pdftitle={Cayley graph of Z * Z}}

\usepackage{letltxmacro}

\usepackage{datetime}
\count255=\time
\multiply\count255 by\year
\advance\count255 by\currentsecond
\edef\randomseed{\the\count255}
%\pgfmathsetseed{\randomseed}  % done later

\usetikzlibrary{calc,arrows,spy}

\tikzset{%
          myemph/.style={#1!70,line width=1.5pt},
          myarrow/.style={pos=0.55},
}

\newcommand{\myarrow}[1][0]{%
\tikz\draw[-triangle 90] (0,0) -- (#1:0.01pt);
}

%%%%%%%%%%%%%%%%%%%%%%%%%%%%%%%%%%%%%%%%%%%%%%%%%%%%%%%%%%%%%%%%%%%%%%%%%%%

\newcount\graphdepth
% We use a TeX counter because LaTeX's counters (\newcounter) are always
% modified globally whereas we need local decrement within each recursion
% branch.

% Set \curLen and \recFactor prior to calling \drawGraph!
%
% The generated graph is centered at (0,0) and its maximal width and height
% are both 2*\curLen. The number 0 < \recFactor <= 0.5 specifies the next
% branch point of the graph: at (1-\recFactor)*\curLen distance from the
% last branch point (which is the origin in the first recursion step).
% Also \curLen of the next recursion step will be \recFactor*\curLen.
%
\newcommand\drawGraph[1]{ % \drawGraph{recursion depth}
    \graphdepth #1\relax
    \path (0,0) coordinate (cur\the\graphdepth);
    \drawGraphElement[180,90,0,-90]
}

\newcommand\drawGraphElement[1][90,0,-90]{%
    % \drawGraphElement[directions in degrees, comma separated]
    %
    \ifnum\graphdepth > 0% proceed in recursion
        \foreach \angle in {#1}
        {
            \begin{scope}[rotate=\angle]
                % draw current piece up to the next branch point
                \pgfmathparse{\curLen*(1-\recFactor)}
                \draw (cur\the\graphdepth) -- +(\pgfmathresult,0) 
                    coordinate (tmp) ;
                %
                % decrement recursion level
                \advance\graphdepth -1\relax
                % update current branch point
                \path (tmp) coordinate (cur\the\graphdepth) ;
                % possibly adjust \recFactor (the default is no adjustment)
                \adjustRecFactor
                % recalculate piece length
                \pgfmathsetmacro\curLen{\curLen*\recFactor}
                %
                % recurse
                \drawGraphElement
            \end{scope}
        
        }
    \else % recursion ends
        % draw last part of the current piece
        \draw (cur\the\graphdepth) -- +(\curLen,0) ;
    \fi
}

\let\adjustRecFactor=\relax

%%%%%%%%%%%%%%%%%%%%%%%%%%%%%%%%%%%%%%%%%%%%%%%%%%%%%%%%%%%%%%%%%%%%%%%%%%%

\begin{document}
How \verb|\drawGraph{n}| unfolds:
\begin{center}
    %
    \pgfmathsetmacro\curLen{1.5}
    \pgfmathsetmacro\recFactor{1.0/3.0}
    %
    \begin{tikzpicture}
        \foreach \j in {1,2,3}
        {
            \begin{scope}[shift={(2.75*\curLen*\j,0)}]
                \expandafter\drawGraph \j
                \path (0,\curLen) ++(0,0.5) node {$n=\j$} ;
            \end{scope}
        }
    \end{tikzpicture}
\end{center}

\vspace{1cm}

\verb|\drawGraph{5}| with highlighted parts and demonstration of
Ti\textit{k}Z's \texttt{spy} library:

\begin{center}
    %
    \pgfmathsetmacro\curLen{5.0}
    \pgfmathsetmacro\recFactor{1.0/2.5}
    %
    \begin{tikzpicture}[%
        spy using overlays={circle, magnification=3, size=5cm}
    ]
        %
        \begin{scope}
            \pgfmathparse{\curLen*(1-\recFactor)}
            \coordinate (A) at (-\pgfmathresult,0);
            \coordinate (B) at ( \pgfmathresult,0);
            \coordinate (C) at (0,-\pgfmathresult);
            \coordinate (D) at (0, \pgfmathresult);
            
            \foreach \from/\to in {{(A)/(0,0)},{(0,0)/(B)}}
            {
                \draw [myemph=red] \from--\to
                    node [myarrow,label={above:a}] {\myarrow[0]} ;
            }
            \foreach \from/\to in {{(C)/(0,0)},{(0,0)/(D)}}
            {
                \draw [myemph=blue] \from--\to
                    node [myarrow,label={right:b}] {\myarrow[90]} ;
            }
        \end{scope}
        
        \coordinate (magni) at
            ($ {\curLen*(1-\recFactor)}*(1,0)%
              +{\curLen*\recFactor*(1-\recFactor)}*(0,1)$) ;
              
        \begin{scope}[shift={(magni)}]
            \pgfmathparse{\curLen*(\recFactor^2)*(1-\recFactor)}
            \coordinate (A) at (-\pgfmathresult,0);
            \coordinate (B) at ( \pgfmathresult,0);
            \coordinate (D) at (0, \pgfmathresult);
            \pgfmathparse{\curLen*\recFactor*(1-\recFactor)}
            \coordinate (C) at (0,-\pgfmathresult);
            
            \foreach \from/\to in {{(A)/(0,0)},{(0,0)/(B)}}
            {
                \draw [myemph=red] \from--\to
                    node [pos=0.65] {\myarrow[0]} ;
            }
            \foreach \from/\to in {{(C)/(0,0)},{(0,0)/(D)}}
            {
                \draw [myemph=blue] \from--\to
                    node [pos=0.75] {\myarrow[90]} ;
            }
        \end{scope}
        
        \drawGraph{5}
        
        \spy [yellow] on (magni) in node at (\curLen*1.3,0.7*\curLen) ;
    \end{tikzpicture}
\end{center}

\vfill
The next page has the width of DIN~A2 paper and shows a graph
at recursion level~9:

\newpage
\KOMAoptions{paper=42cm:44cm,pagesize}
\recalctypearea
%
\vspace*{-1.5cm}
\centering
\makebox[0pt]{%
    %
    \pgfmathsetmacro\curLen{0.97} % (note that we set 'x' and 'y' of the
                                  % tikzpicture environment to (almost)
                                  % half the page width)
    \pgfmathsetmacro\recFactor{0.485}
    % to achieve a better result we decrease the \recFactor with
    % each recursion step:
    \def\adjustRecFactor{%
        \pgfmathsetmacro\recFactor{\recFactor-0.01}}
    %
    \begin{tikzpicture}[x=0.44\paperwidth, y=0.44\paperwidth]
        \drawGraph{9}
        %\drawGraph{1}
    \end{tikzpicture}%
}

\let\adjustRecFactor=\relax

%%%%%%%%%%

\newpage
\KOMAoptions{paper=a4,pagesize}
\recalctypearea
%
And just for fun: some ``random art'' via \verb|\drawGraphElement|~\ldots
\begin{center}
    \pgfmathsetmacro\curLen{5}
    \pgfmathsetmacro\recFactor{0.4}
    \LetLtxMacro{\dGE}{\drawGraphElement}
    %
    \newcommand{\randomAngles}[1]{%
        \begingroup%
        \pgfmathsetmacro\n{random(#1)}%
        \gdef\angles##1{}%
        \foreach \j in {1,...,\n} {%
            \pgfmathparse{random(0,359)}\xdef\angles{\angles,\pgfmathresult}}%
        \endgroup%
    }
    %
    \newcommand{\RR}{%
        \begin{tikzpicture}
            \randomAngles{2,10} % generates 2--10 random angles
            \pgfmathsetmacro\depth{random(2,4)}
            \expandafter\graphdepth\depth\relax
            \edef\zz{\unexpanded{\renewcommand\drawGraphElement}%
                {\noexpand\dGE[\angles]}%
            }\zz
            \path (0,0) coordinate (cur\the\graphdepth);
            \drawGraphElement
        \end{tikzpicture}%
    }
    
    \vfill
    
    \pgfmathsetseed{37}
    \RR
    
    \vfill
    
    \pgfmathsetseed{144}
    \RR
    \hfill
    \pgfmathsetseed{63}
    \raisebox{1.5cm}{\RR}
    \hfill
    \pgfmathsetseed{48}
    \RR
    
    \newpage
    \pgfmathsetseed{\randomseed}
    %
    \RR
    
    \vfill
    
    \RR
\end{center}

\end{document}
