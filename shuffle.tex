\documentclass{tixtex}
\hypersetup{pdftitle={shuffles}}

\tikzset{%
    Dline/.style={line width=2pt, color=blue},
    Dgrid/.style={line width=0.3pt, color=black!40}
}

%%%%%%%%%%%%%%%%%%%%%%%%%%%%%%%%%%%%%%%%%%%%%%%%%%%%%%%%%%%%%%%%%%%%%%%%%%%

% internal layer
%%%%%%%%%%%%%%%%

% \shuf{current link}{steps left to take [U]p}{steps leftto take [R]ight}
%
\newcommand{\shuf}[3]{%  
    \ifnum#2 < 1 % no more Us
        \ifnum#3 < 1 % no more Rs
            % recursion ends, add link to the 'list' of links
            \xdef\links{%
                \unexpanded\expandafter{\links}%
                \noexpand\LINK\unexpanded{#1}\noexpand\ENDLINK%
            }
        \else % more Rs
            \shuf{\RR#1}{0}{\numexpr#3-1\relax}
        \fi
    \else % more Us
        \ifnum#3 < 1 % no more Rs
            \shuf{\UU#1}{\numexpr#2-1\relax}{0}
        \else % more Rs
            % 'normal' recursion step:
            % recurse in both directions
            \shuf{\UU#1}{\numexpr#2-1\relax}{#3}
            \shuf{\RR#1}{#2}{\numexpr#3-1\relax}
        \fi
    \fi
}

\def\processlink\LINK#1\ENDLINK{%
    \ifx\NOMORELINKS#1
        \relax
    \else
        \drawlink{#1}
        \xdef\ithlink{\numexpr\ithlink+1\relax}
        \expandafter\processlink
    \fi
}

% graphical layer
%%%%%%%%%%%%%%%%%
\newcommand{\RR}{-- ++(0:\stepsize)}
\newcommand{\UU}{-- ++(90:\stepsize)}

\newcommand{\drawlink}[1]{
    \pgfmathsetlengthmacro{\xshiftwidth}{%
        mod(\ithlink,\cols)*(\xdim*\stepsize+\xspacing)%
    }
    \pgfmathsetlengthmacro{\yshiftheight}{%
        floor(\ithlink/\cols)*(\ydim*\stepsize+\yspacing)%
    }
    %
    \begin{scope}[xshift=\xshiftwidth,yshift=-\yshiftheight]
        \draw [step=\stepsize,Dgrid] (0,0) grid (\xdim*\stepsize,\ydim*\stepsize);
        \draw [Dline] (0,0)
            #1
        ;
    \end{scope}
}

% frontend layer
%%%%%%%%%%%%%%%%

% Set the macros
%     \width      total (maximal) width of a row
%     \stepsize   length of a single horizontal or vertical step
%     \xspacing   gap size between each lattice in the same row
%     \yspacing   gap size between rows
% prior to calling \shuffle. (See below for examples.)
%
% \shuffle{lattice height}{lattice width}
%
\newcommand{\shuffle}[2]{
    \gdef\ydim{#1}\gdef\xdim{#2}
    \gdef\links{}
    \shuf{}{#1}{#2}
    \gdef\ithlink{0}
    \pgfmathtruncatemacro{\cols}{%
          (\width - \xdim*\stepsize)    %
        / (\xdim*\stepsize + \xspacing) %
        + 1                             %
    }
    \expandafter\processlink\links\LINK\NOMORELINKS\ENDLINK
}

%%%%%%%%%%%%%%%%%%%%%%%%%%%%%%%%%%%%%%%%%%%%%%%%%%%%%%%%%%%%%%%%%%%%%%%%%%%

\begin{document}
A $(p,q)$-shuffle is a shortest path through a $p\times q$ lattice
(from one edge to the diagonally opposite one where steps are allowed
horizontally and vertically).\footnote{%
    \url{http://ncatlab.org/nlab/show/shuffle}}
The macro \verb|\shuffle| draws all shuffles through a specified lattice,
e.\,g. \verb|\shuffle{2}{2}| and \verb|\shuffle{2}{3}|:

\vspace{1cm}

\begin{center}
    \makebox[\textwidth]{%
    \begin{tikzpicture}
        \newcommand{\width}{1.2\textwidth}
        \newcommand{\stepsize}{1cm}
        \newcommand{\xspacing}{1.0cm}
        \newcommand{\yspacing}{0pt}
        %
        \shuffle{2}{2}
    \end{tikzpicture}%
    }

    \vspace{2cm}

    \begin{tikzpicture}
        \newcommand{\width}{1.03\textwidth}
        \newcommand{\stepsize}{1cm}
        \newcommand{\xspacing}{1.0cm}
        \newcommand{\yspacing}{0.8cm}
        %
        \shuffle{2}{3}
    \end{tikzpicture}
\end{center}

\newpage
This page shows all shuffles through a $4\times 4$ lattice:

\vspace{0.5cm}

\begin{center}
    \begin{tikzpicture}
        \newcommand{\width}{\textwidth}
        \newcommand{\stepsize}{3mm}
        \newcommand{\xspacing}{1.0cm}
        \newcommand{\yspacing}{0.8cm}
        %
        \shuffle{4}{4}
    \end{tikzpicture}
\end{center}
\end{document}
